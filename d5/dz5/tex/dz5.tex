

\def\qtr{јесен 2021.}
\def\due{понедељак, 27. децембар у 23:59}
\def\psetnum{5}

%%% Change the following flag to toggle between questions or solutions
\ifdefined\solutions {} \else \def\solutions{1} \fi


\documentclass{article}

\usepackage[utf8]{inputenc}
%\usepackage[T1,T2A]{fontenc}
\usepackage[T2A]{fontenc}
\usepackage[serbian]{babel}

\usepackage{graphicx}

\newcommand{\di}{{d}}
\newcommand{\nexp}{{n}}
\newcommand{\nf}{{p}}
\newcommand{\vcd}{{\textbf{D}}}

\usepackage{nccmath}
\usepackage{mathtools}
\usepackage{graphicx,caption}
\usepackage{enumitem}
\usepackage{epstopdf,subcaption}
\usepackage{psfrag}
\usepackage{amsmath,amssymb,epsf}
\usepackage{verbatim}
\usepackage[hyphens]{url}
\usepackage{color}
\usepackage{bbm}
\usepackage{listings}
\usepackage{setspace}
\usepackage{float}
\usepackage{natbib}
\definecolor{Code}{rgb}{0,0,0}
\definecolor{Decorators}{rgb}{0.5,0.5,0.5}
\definecolor{Numbers}{rgb}{0.5,0,0}
\definecolor{MatchingBrackets}{rgb}{0.25,0.5,0.5}
\definecolor{Keywords}{rgb}{0,0,1}
\definecolor{self}{rgb}{0,0,0}
\definecolor{Strings}{rgb}{0,0.63,0}
\definecolor{Comments}{rgb}{0,0.63,1}
\definecolor{Backquotes}{rgb}{0,0,0}
\definecolor{Classname}{rgb}{0,0,0}
\definecolor{FunctionName}{rgb}{0,0,0}
\definecolor{Operators}{rgb}{0,0,0}
\definecolor{Background}{rgb}{0.98,0.98,0.98}
\lstdefinelanguage{Python}{
numbers=left,
numberstyle=\footnotesize,
numbersep=1em,
xleftmargin=1em,
framextopmargin=2em,
framexbottommargin=2em,
showspaces=false,
showtabs=false,
showstringspaces=false,
frame=l,
tabsize=4,
% Basic
basicstyle=\ttfamily\footnotesize\setstretch{1},
backgroundcolor=\color{Background},
% Comments
commentstyle=\color{Comments}\slshape,
% Strings
stringstyle=\color{Strings},
morecomment=[s][\color{Strings}]{"""}{"""},
morecomment=[s][\color{Strings}]{'''}{'''},
% keywords
morekeywords={import,from,class,def,for,while,if,is,in,elif,else,not,and,or
,print,break,continue,return,True,False,None,access,as,,del,except,exec
,finally,global,import,lambda,pass,print,raise,try,assert},
keywordstyle={\color{Keywords}\bfseries},
% additional keywords
morekeywords={[2]@invariant},
keywordstyle={[2]\color{Decorators}\slshape},
emph={self},
emphstyle={\color{self}\slshape},
%
}


\pagestyle{empty} \addtolength{\textwidth}{1.0in}
\addtolength{\textheight}{0.5in}
\addtolength{\oddsidemargin}{-0.5in}
\addtolength{\evensidemargin}{-0.5in}
\newcommand{\ruleskip}{\bigskip\hrule\bigskip}
\newcommand{\nodify}[1]{{\sc #1}}
\newcommand{\points}[1]{{\textbf{[#1 поена]}}}
\newcommand{\subquestionpoints}[1]{{[#1 поена]}}
\newenvironment{answer}{{\bf Одговор:} \sf \begingroup\color{red}}{\endgroup}%

\newcommand{\bitem}{\begin{list}{$\bullet$}%
{\setlength{\itemsep}{0pt}\setlength{\topsep}{0pt}%
\setlength{\rightmargin}{0pt}}}
\newcommand{\eitem}{\end{list}}

\setlength{\parindent}{0pt} \setlength{\parskip}{0.5ex}
\setlength{\unitlength}{1cm}

\renewcommand{\Re}{{\mathbb R}}
\newcommand{\R}{\mathbb{R}}
\newcommand{\what}[1]{\widehat{#1}}

\renewcommand{\comment}[1]{}
\newcommand{\mc}[1]{\mathcal{#1}}
\newcommand{\half}{\frac{1}{2}}

\def\KL{D_{KL}}
\def\xsi{x^{(i)}}
\def\ysi{y^{(i)}}
\def\zsi{z^{(i)}}
\def\E{\mathbb{E}}
\def\calN{\mathcal{N}}
\def\calD{\mathcal{D}}

\usepackage{tikz}
\usepackage{bbding}
\usepackage{pifont}
\usepackage{wasysym}
\usepackage{amssymb}
\usepackage{booktabs}
\usepackage{verbatim}



\def\shownotes{0}  %set 1 to show author notes
\ifnum\shownotes=1
\newcommand{\authnote}[2]{$\ll$\textsf{\footnotesize #1 notes: #2}$\gg$}
\else
\newcommand{\authnote}[2]{}
\fi

\newcommand{\tnote}[1]{{\color{blue}\authnote{Влада}{#1}}}
\newcommand{\fk}[1]{{\color{purple}{[FK:#1]}}}
\newcommand{\notes}[1]{{\color{blue} Note:} \textit{#1} \newline}


\begin{document}

\pagestyle{myheadings} \markboth{}{Основи машинског учења --- домаћи задатак \textnumero\psetnum}

\ifnum\solutions=1{
{\huge\noindent Основи машинског учења, \qtr\\
домаћи задатак \textnumero\psetnum\;решења}\\\\
ИМЕ И ПРЕЗИМЕ (\texttt{БРОЈ ИНДЕКСА})
} \else {\huge
\noindent Основи машинског учења, \qtr\\
домаћи задатак \textnumero\psetnum
} \fi

\ruleskip

{\bf Рок: {\due } на Moodle-у.}

\medskip

{\bf Упутства:} (1) Ова питања захтевају размишљање, не и дуге одговоре. Будите што сажетији. (2) Уколико има било каквих нејасноћа, питајте предметног наставника или сарадника. (3) Студенти могу радити и послати решења самостално или у паровима. У случају заједничког рада, имена и презимена оба студента морају бити назначена у Gradescope-у и није дозвољено радити са истим колегом више од једном. (4) За програмерске задатке, коришћење напредних библиотека за машинско учење попут scikit-learn није дозвољено. (5) Кашњење приликом слања односно свака пошиљка након рока носи негативне поене.



\smallskip

Сви студенти морају послати електронску PDF верзију својих решења. Препоручено је куцање одговора у \LaTeX-у које са собом носи 10 додатних поена. Сви студенти такође морају на Moodle-у послати и zip датотеку која садржи изворни код, а коју би требало направити користећи \texttt{make\_zip.py} скрипту. Обавезно (1) користити само стандардне библиотеке или оне које су већ учитане у шаблонима и (2) осигурати да се програми извршавају без грешки. Послати изворни код може бити покретан од стране аутоматског оцењивача над унапред недоступним скупом података за тестирање, али и коришћен за верификацију излаза који су дати у извештају.


\vfill

{\bf Кодекс академске честитости:} Иако студенти могу радити у паровима, није дозвољена сарадња на изради домаћих задатака у ширим групама. Изричито је забрањено било какво дељење одговора. Такође, копирање решења са интернета није дозвољено. Свако супротно поступање сматра се тешком повредом академске честитости и биће најстроже кажњено.




\begin{enumerate}[wide, labelwidth=!, labelindent=0pt]

\clearpage
{\bf Упутства:} (1) Ова питања захтевају размишљање, не и дуге одговоре. Будите што сажетији. (2) Уколико има било каквих нејасноћа, питајте предметног наставника или сарадника. (3) Студенти могу радити и послати решења самостално или у паровима. У случају заједничког рада, имена и презимена оба студента морају бити назначена у Gradescope-у и није дозвољено радити са истим колегом више од једном. (4) За програмерске задатке, коришћење напредних библиотека за машинско учење попут scikit-learn није дозвољено. (5) Кашњење приликом слања односно свака пошиљка након рока носи негативне поене.



\smallskip

Сви студенти морају послати електронску PDF верзију својих решења. Препоручено је куцање одговора у \LaTeX-у које са собом носи 10 додатних поена. Сви студенти такође морају на Moodle-у послати и zip датотеку која садржи изворни код, а коју би требало направити користећи \texttt{make\_zip.py} скрипту. Обавезно (1) користити само стандардне библиотеке или оне које су већ учитане у шаблонима и (2) осигурати да се програми извршавају без грешки. Послати изворни код може бити покретан од стране аутоматског оцењивача над унапред недоступним скупом података за тестирање, али и коришћен за верификацију излаза који су дати у извештају.


\vfill

{\bf Кодекс академске честитости:} Иако студенти могу радити у паровима, није дозвољена сарадња на изради домаћих задатака у ширим групама. Изричито је забрањено било какво дељење одговора. Такође, копирање решења са интернета није дозвољено. Свако супротно поступање сматра се тешком повредом академске честитости и биће најстроже кажњено.




\end{enumerate}

\end{document}
